\documentclass[a4paper,10pt,openany,oneside]{report}

\usepackage[utf8]{inputenc}
\usepackage[T1]{fontenc}
\usepackage[francais]{babel}
\usepackage{eurosym}
\usepackage[margin=1.5in]{geometry}
\usepackage{graphicx}
\usepackage{fancyhdr}
\usepackage{xcolor}

\pagestyle{fancy}
\renewcommand{\headrulewidth}{0pt}

\setlength{\oddsidemargin}{0pt}
\setlength{\topmargin}{0pt}
\setlength{\marginparwidth}{0pt}
\setlength{\headheight}{30pt}
\setlength{\headwidth}{450pt}
\setlength{\headsep}{10pt}
\setlength{\voffset}{0pt}
\setlength{\hoffset}{0pt}
\setlength{\footskip}{70pt}
\setlength{\textwidth}{450pt}
\setlength{\textheight}{580pt}

\title{SYR DeaDBeeF - Projet SYR2 - Compte Rendu}
\author{Antoine \bsc{Pinsard}}
\date{8 Mars 2015}

\makeatletter
\lhead{\Large{\bf\@title}}
\lfoot{\footnotesize \color[gray]{0.5} \@title}
\rfoot{\footnotesize \color[gray]{0.5} \@author / \@date}
\makeatother

\renewcommand{\thesection}{\arabic{section}}

\begin{document}

\maketitle

\section{Un lecteur audio}

\subsection{Que se passe-t-il si on déclare une fausse fréquence
            d'échantillonnage à la sortie audio ?}

Lorsqu'on déclare une fréquence d'échantillonnage 2 fois plus grande, la
vitesse de lecture du son est multipliée par 2. Inversement, lorsqu'on déclare
une fréquence d'échantillonnage 2 fois plus petite, le fichier son est joué
deux fois plus lentement.

Plus la fréquence d'échantillonage est élevée, plus on a capturé d'informations
sur le son dans un interval de temps donné. Si on renseigne une fréquence x
fois plus élevée que la fréquence réelle, on va lire x fois plus d'informations
dans une même période de temps. Ce qui explique le phénomène observé.

\subsection{Que se passe-t-il si vous déclarez à la sortie audio que le fichier
            est mono ?}

Si on déclare que le fichier est en mono alors qu'il est en stereo, le son est
plus lent et plus grave. C'est le même effet que lorsqu'on lit une cassette audio magnetique avec une vitesse de rotation plus faible que la normale.

\subsection{Que se passe-t-il si vous déclarez à la sortie audio une mauvaise
            taille d'échantillons ?}

Ça fait sursauter, enlever vite instinctivement les oreillettes, et baisser le
volume avant de remettre les écouteurs craintivement l'un après l'autre. On
entend alors un faible "pshhhh" dans l'oreille gauche et on parvient tout de
même à reconnaître la mélodie, jouée par un piano Fisher-Price dont les piles
sont en fin de vie, dans l'oreille droite.

\end{document}
